Backpropagation metodas


\begin{equation*} \label{eq:Et}
  \begin{aligned}
    \vec{E} = &\left( x^2 \right) = \\ % komentaras
    &b+\sqrt{c} + h + 1 \\
    a^{(t)} = \left \{
    \begin{aligned}
      &\text{kai } l=1 & d+f\\
      &\text{kai } l=2 & d+f+8
    \end{aligned}
    \right.
  \end{aligned}
\en{equation*}



\begin{equation} \label{eq:Ett}
  \begin{aligned}
  E^{(t)} = \sum_{k=1}^{M} \frac{1}{2}(y_k^{(t)} - h_k^{(t)})^{2}
  \end{aligned}
\end{equation}

\begin{equation} \label{eq:E_deriv}
  \begin{aligned}
  \frac{\partial E^{(t)}}{\partial w_{ij}^{(v,s)}} = \sum_{n=1}^{M} \frac{\partial (\sum_{k=1}^{M} \frac{1}{2}(y_k^{(t)} - h_k^{(t)})^{2})}{\partial h_n^{(t)}} * \frac{\partial h_n^{(t)}}{\partial w_{ij}^{(v,s)}}
  \end{aligned}
\end{equation}


\begin{equation} \label{eq:hn}
  \begin{aligned}
  \frac{\partial h_k^{(t)}}{\partial w_{ij}^{(v,s)}}
  =
  \sum_{n=1}^{M}
  \frac{\partial (\frac{e^{b_k^{(t)}}}{\sum_{m=1}^{M} e^{b_m^{(t)}}})}
  {\partial b_n^{(t)}}
  *
   \frac{\partial b_n^{(t)}}{\partial w_{ij}^{(v,s)}}
   \end{aligned}
\end{equation}

\begin{equation} \label{eq:bn}
  \begin{aligned}
  \frac{\partial b_n^{(t)}}{\partial w_{ij}^{(v,s)}}
  =
  \frac{\partial (a_k^{(o,L)} tanh(c_k^{(t)}))}{\partial w_{ij}^{(v,s)}}
  =
  \frac{\partial a_k^{(o,L)}}{\partial w_{ij}^{(v,s)}} tanh(c_k^{(t)}) +
  a_k^{(o,L)} * \frac{\partial tanh(c_k^{(t)}}{\partial w_{ij}^{(v,s)}}
  \end{aligned}
\end{equation}

Isskiriam dvi isvestines


\begin{equation} \label{eq:tanh_deriv_byc}
  \begin{aligned}
  \frac{\partial tanh(c_k^{(t)}}{\partial w_{ij}^{(v,s)}} =
  \frac{\partial tanh(c_k^{(t)}}{\partial c_k^{(t)}}
  \frac{\partial c_k^{(t)}}{\partial w_{ij}^{(v,s)}}
  \end{aligned}
\end{equation}


\begin{equation} \label{eq:c_deriv}
  \begin{aligned}
  \frac{\partial c_k^{(t)}}{\partial w_{ij}^{(v,s)}} =
  \frac{\partial (c_k^{(t-1)}*a_k^{(f,L)}+a_k^{(i,L)}*a_k^{(g,L)})}{\partial w_{ij}^{(v,s)}} =\\
  \frac{ \partial c_k^{(t-1)}}{\partial w_{ij}^{(v,s)}}* a_k^{(f,L)} +
  c_k^{(t-1)} * \frac{\partial a_k^{(f,L)}}{\partial w_{ij}^{(v,s)}} +
  \frac{\partial a_k^{(i,L)}}{\partial w_{ij}^{(v,s)}} * a_k^{(g,L)} +
  a_k^{(i,L)} * \frac{\partial a_k^{(g,L)}}{\partial w_{ij}^{(v,s)}}
  \end{aligned}
\end{equation}


toliau isvesima bendraja a isvestines formule pagal bet kokio tinklo svorius

\begin{equation} \label{eq:a_derivl-1}
  \begin{aligned}
  \frac{\partial a_k^{(u, L)}}{\partial w_{ij}^{(v,L-1)}} =
  \frac{\partial f(z_k^{(u, L)})}{\partial w_{ij}^{(v,L-1)}} =
  \frac{\partial f(z_k^{(u, L)})}{\partial z_k^{(u,L)}} *
  \frac{\partial z_k^{(u,L)}}{\partial w_{ij}^{(v,L-1)}} \\
  \end{aligned}
\end{equation}

\begin{equation} \label{eq:z_deriv}
  \begin{aligned}
    \frac{\partial z_k^{(u,L)}}{\partial w_{ij}^{(v,L-1)}} = \\
    \sum_{n=1}^{K(u, L-1)+1} \frac{\partial (\sum_{m=1}^{K(u,L-1)+1} w_{mk}^{(u,L-1) * a_m^{(u,L-1)}} )}{\partial a_n^{(u,L-1)}} *
    \frac{\partial a_n^{(u,L-1)}}{\partial w_{ij}^{(v,L-1)}}   +\\
    \sum_{n=1}^{K(u, L-1)+1} \frac{\partial (\sum_{m=1}^{K(u,L-1)+1} w_{mk}^{(u,L-1) * a_m^{(u,L-1)}} )}{\partial w_{nk}^{(v,L-1)}} *
    \frac{\partial w_{nk}^{(v,L-1)}}{\partial w_{ij}^{(v,L-1)}} = \\
    \sum_{n=1}^{K(u, L-1)+1} w_{nk}^{(u,L-1)} *   \frac{\partial a_k^{(u, L-1)}}{\partial w_{ij}^{(v,L-1)}}  +  \frac{\partial f(z_k^{(u, L)})}{\partial z_k^{(u,L)}} * \delta_{u,v} a_i^{(u,L-1)}
  \end{aligned}
\end{equation}


Istacius dar viena a isvestine gauname

kai

\begin{equation} \label{eq:plius1nereikia}
  \begin{aligned}
  \text{Kai n = K(u,L-1)+1, tai tada yra skaičiuojama Bias neurono išvestinė, pagal svorį} \\(t.y.\\
    \partial a_{K(u,L-1)+1}^{(u,L-1)}}{\partial w_{ij}^{(v,L-1)}
  ). Kadangi, Bias neurono reikšmė nepriklauso nuo svorių reikšmių, tuomet Bias neurono išvestinė pagal bet kokį svorį yra lygi 0(
    \partial a_{K(u,L-1)+1}^{(u,L-1)}}{\partial w_{ij}^{(v,L-1)} =
    \partial 1}{\partial w_{ij}^{(v,L-1)} = 0
  ). Tai yra esminis momentas, norint greitinti skaičiavimus optimizuojant programą.
\end{aligned}
\end{equation}

\begin{equation}
  \begin{aligned}
  Toliau skaičiuojant \frac{\partial a_k^{(u, L)}}{\partial w_{ij}^{(v,L-1)}} išvestinę ir įsistačius žemesnio sluoksnio a išvestinę į lygybę ( \frac{\partial a_k^{(u, L-1)}}{\partial w_{ij}^{(v,L-1)}} gauname lygybę .
  \end{aligned}
\end{equation}
% Toliau skaičiuojant \frac{\partial a_k^{(u, L)}}{\partial w_{ij}^{(v,L-1)}} išvestinę ir įsistačius žemesnio sluoksnio a išvestinę į lygybę (\frac{\partial a_k^{(u, L-1)}}{\partial w_{ij}^{(v,L-1)}}) gauname lygybę .

\begin{equation} \label{eq:a_deriv_2}
  \begin{aligned}
  \frac{\partial a_k^{(u, L)}}{\partial w_{ij}^{(v,L-1)}} = \\
  \frac{\partial f(z_k^{(u, L)})}{\partial z_k^{(u,L)}} *
  \sum_{n=1}^{K(u, L-1)} w_{nk}^{(u,L-1)} *
  \frac{\partial f(z_n^{(u, L-1)})}{\partial z_n^{(u,L-1)}}
  *
  \sum_{p=1}^{K(u, L-2)} w_{pn}^{(u,L-2)} *
  \frac{\partial a_p^{(u,L-2)}}{\partial w_{ij}^{(v,L-1)}} +
  \frac{\partial f(z_k^{(u, L)})}{\partial z_k^{(u,L)}} * \delta_{u,v}a_i^{(u,L-1)} \\
  pakeičiame skaičiavimų tvarką, taip kad skaičiavimai iš pradžių būtų sumuojami pagal aukštesnio sluoksnio neuronų kiekį, o poto pagal žemesnio. Tada gauname (sdfsf) lygybę:\\
  \sum_{p=1}^{K(u, L-2)}
  \frac{\partial a_p^{(u,L-2)}}{\partial w_{ij}^{(v,L-1)}} *
  \frac{\partial f(z_k^{(u, L)})}{\partial z_k^{(u,L)}}
  *
  \sum_{n=1}^{K(u, L-1)} w_{pn}^{(u,L-2)} * w_{nk}^{(u,L-1)} * \frac{\partial f(z_n^{(u, L-1)})}{\partial z_n^{(u,L-1)}} + \frac{\partial f(z_k^{(u, L)})}{\partial z_k^{(u,L)}} * \delta_{u,v}a_i^{(u,L-1)}
  Šioje lygybėje (asda) įsivedame žymėjimą \\ G_{pk}^{(u,L)} =
  \frac{\partial f(z_k^{(u, L)})}{\partial z_k^{(u,L)}}
  *
  \sum_{n=1}^{K(u, L-1)} w_{pn}^{(u,L-2)} * w_{nk}^{(u,L-1)} * \frac{\partial f(z_n^{(u, L-1)})}{\partial z_n^{(u,L-1)}}
  \end{aligned}
\end{equation}























\begin{equation} \label{eq:a_derivl-2}
  \begin{aligned}
  \frac{\partial a_k^{(u, L)}}{\partial w_{ij}^{(v,L-2)}} =
  \frac{\partial f(z_k^{(u, L)})}{\partial w_{ij}^{(v,L-2)}} =
  \frac{\partial f(z_k^{(u, L)})}{\partial z_k^{(u,L)}} *
  \frac{\partial z_k^{(u,L)}}{\partial w_{ij}^{(v,L-2)}} \\
  \end{aligned}
\end{equation}

\begin{equation} \label{eq:z_deriv-2}
  \begin{aligned}
  \frac{\partial z_k^{(u,L)}}{\partial w_{ij}^{(v,L-2)}} = \\
  \sum_{n=1}^{K(u, L-1)} \frac{\partial (\sum_{m=1}^{K(u,L-1)} w_{mk}^{(u,L-1) * a_m^{(u,L-1)}} )}{\partial a_n^{(u,L-1)}} *
  \frac{\partial a_n^{(u,L-1)}}{\partial w_{ij}^{(v,L-2)}}   +\\
  \sum_{n=1}^{K(u, L-1)+1} \frac{\partial (\sum_{m=1}^{K(u,L-1)+1} w_{mk}^{(u,L-1) * a_m^{(u,L-1)}} )}{\partial w_{nk}^{(v,L-1)}} *
  \frac{\partial w_{nk}^{(v,L-1)}}{\partial w_{ij}^{(v,L-2)}}  \\
  Čia suma yra lygi 0, nes  \frac{\partial w_{nk}^{(v,L-1)}}{\partial w_{ij}^{(v,L-2)}}
  visada bus lygi 0, nes aukštesnio sluoksnio svorio išvestinė, pagal žemesnio sluoksnio svorio išvestinę yra lygu 0.

  \sum_{n=1}^{K(u, L-1)+1} w_{nk}^{(u,L-1)} *   \frac{\partial a_k^{(u, L-1)}}{\partial w_{ij}^{(v,L-1)}}  +  \frac{\partial f(z_k^{(u, L)})}{\partial z_k^{(u,L)}} * \delta_{u,v} a_i^{(u,L-1)}
\end{aligned}
\end{equation}


Istacius dar viena a isvestine gauname

kai

\begin{equation} \label{eq:plius1nereikia}
  \begin{aligned}
  Kai n=K(u,L-1)+1, tai tada yra skaičiuojama Bias neurono išvestinė, pagal svorį (t.y.
    \partial a_{K(u,L-1)+1}^{(u,L-1)}}{\partial w_{ij}^{(v,L-1)}
  ). Kadangi, Bias neurono reikšmė nepriklauso nuo svorių reikšmių, tuomet Bias neurono išvestinė pagal bet kokį svorį yra lygi 0(
    \partial a_{K(u,L-1)+1}^{(u,L-1)}}{\partial w_{ij}^{(v,L-1)} =
    \partial 1}{\partial w_{ij}^{(v,L-1)} = 0
  ). Tai yra esminis momentas, norint greitinti skaičiavimus optimizuojant programą.
\end{aligned}
\end{equation}

\begin{equation}
  \begin{aligned}
  Toliau skaičiuojant \frac{\partial a_k^{(u, L)}}{\partial w_{ij}^{(v,L-1)}} išvestinę ir įsistačius žemesnio sluoksnio a išvestinę į lygybę ( \frac{\partial a_k^{(u, L-1)}}{\partial w_{ij}^{(v,L-1)}} gauname lygybę .
  \end{aligned}
\end{equation}
% Toliau skaičiuojant \frac{\partial a_k^{(u, L)}}{\partial w_{ij}^{(v,L-1)}} išvestinę ir įsistačius žemesnio sluoksnio a išvestinę į lygybę (\frac{\partial a_k^{(u, L-1)}}{\partial w_{ij}^{(v,L-1)}}) gauname lygybę .

\begin{equation} \label{eq:a_deriv_2}
  \begin{aligned}
  \frac{\partial a_k^{(u, L)}}{\partial w_{ij}^{(v,L-1)}} = \\
  \frac{\partial f(z_k^{(u, L)})}{\partial z_k^{(u,L)}} *
  \sum_{n=1}^{K(u, L-1)} w_{nk}^{(u,L-1)} *
  \frac{\partial f(z_n^{(u, L-1)})}{\partial z_n^{(u,L-1)}}
  *
  \sum_{p=1}^{K(u, L-2)} w_{pn}^{(u,L-2)} *
  \frac{\partial a_p^{(u,L-2)}}{\partial w_{ij}^{(v,L-1)}} +
  \frac{\partial f(z_k^{(u, L)})}{\partial z_k^{(u,L)}} * \delta_{u,v}a_i^{(u,L-1)} \\
  pakeičiame skaičiavimų tvarką, taip kad skaičiavimai iš pradžių būtų sumuojami pagal aukštesnio sluoksnio neuronų kiekį, o poto pagal žemesnio. Tada gauname (sdfsf) lygybę:\\
  \sum_{p=1}^{K(u, L-2)}
  \frac{\partial a_p^{(u,L-2)}}{\partial w_{ij}^{(v,L-1)}} *
  \frac{\partial f(z_k^{(u, L)})}{\partial z_k^{(u,L)}}
  *
  \sum_{n=1}^{K(u, L-1)} w_{pn}^{(u,L-2)} * w_{nk}^{(u,L-1)} * \frac{\partial f(z_n^{(u, L-1)})}{\partial z_n^{(u,L-1)}} + \frac{\partial f(z_k^{(u, L)})}{\partial z_k^{(u,L)}} * \delta_{u,v}a_i^{(u,L-1)}
  Šioje lygybėje (asda) įsivedame žymėjimą \\ G_{pk}^{(u,L)} =
  \frac{\partial f(z_k^{(u, L)})}{\partial z_k^{(u,L)}}
  *
  \sum_{n=1}^{K(u, L-1)} w_{pn}^{(u,L-2)} * w_{nk}^{(u,L-1)} * \frac{\partial f(z_n^{(u, L-1)})}{\partial z_n^{(u,L-1)}}
  \end{aligned}
\end{equation}



\begin{equation*} \label{eq:gkvh}
  \begin{aligned}
    \frac{\partial a_k^{(u,l)}}{\partial w_{ij}^{(v,s)}} = \left \{
    \begin{aligned}
      &\text{kai } l=1 & \frac{\partial h_k^{(t-1)}}{\partial w_{ij}^{(v,s)}}\\
      &\text{kai } l=2  \frac{f(z_k^{(u,l)})}{z_k^{(u,l)}}
      *
      \sum_{n=0}^M w_{nk}^u * \frac{\partial h_n^{(t-1)}}{\partial w_{ij}^{(v,s)}} +
      \delta_{u,v}\frac{f(z_k^{(u,l)})}{z_k^{(u,l)}}a_i^{(u,l-1)}\\
      &\text{kai } l>= 3 \sum_{p=1}^{K(u,l-2)}
      \frac{\partial a_p^{(u,l-2)}}{\partial w_{ij}^{(v,s)}}G_{pk}^{(u,l)} +
      \delta_{l,s+1}\delta_{u,v}\frac{f(z_k^{(u,l)})}{z_k^{(u,l)}}a_i^{(u,l-1)}\\
      \delta_{l,s+2}\delta_{u,v}\frac{f(z_k^{(u,l)})}{z_k^{(u,l)}}a_i^{(u,l-2)}*\sum_{n=1}^{(K(u,l-1)+1)} w_{nk}^{(u,l-1)}\frac{f(z_k^{(u,l-1)})}{z_k^{(u,l-1)}}
    \end{aligned}
    \right.
  \end{aligned}
\en{equation*}




\begin{equation*} \label{eq:gkv}
  \begin{aligned}
    \frac{\partial a_k^{(u,l)}}{\partial w_{ij}^{(v)}} = \left \{
    \begin{aligned}
      &\text{kai } l=1 & \frac{\partial h_k^{(t-1)}}{\partial w_{ij}^{(v)}}\\
      &\text{kai } l=2  \frac{f(z_k^{(u,l)})}{z_k^{(u,l)}}
      *
      \sum_{n=0}^M w_{nk}^u * \frac{\partial h_n^{(t-1)}}{\partial w_{ij}^{(v)}} +
      \delta_{u,v}\frac{f(z_k^{(u,l)})}{z_k^{(u,l)}}h_i^{(t-1)}\\
      &\text{kai } l= 3 \sum_{p=1}^{M}
      \frac{\partial a_p^{(u,l-2)}}{\partial w_{ij}^{(v)}}G_{pk}^{(u,l)} +
      \delta_{l,1}\delta_{u,v}\frac{f(z_k^{(u,l)})}{z_k^{(u,l)}}h_i^{(t-1)}*\sum_{n=1}^{K(u,l-1)+1} w_{nk}^{(u,l-1)}\frac{f(z_k^{(u,l-1)})}{z_k^{(u,l-1)}}
      &\text{kai } l> 3 \sum_{p=1}^{K(u,l-2)}
      \frac{\partial a_p^{(u,l-2)}}{\partial w_{ij}^{(v)}}G_{pk}^{(u,l)} +
      \delta_{l,1}\delta_{u,v}\frac{f(z_k^{(u,l)})}{z_k^{(u,l)}}a_i^{(u,l-2)}*\sum_{n=1}^{K(u,l-1)+1} w_{nk}^{(u,l-1)}\frac{f(z_k^{(u,l-1)})}{z_k^{(u,l-1)}}
    \end{aligned}
    \right.
  \end{aligned}
\en{equation*}
%
% \begin{equation} \label{eq:}
%
% \end{equation}

%
% \begin{equation} \label{eq:}
%
% \end{equation}

%
% \begin{equation} \label{eq:}
%
% \end{equation}

%
% \begin{equation} \label{eq:}
%
% \end{equation}

%
% \begin{equation} \label{eq:}
%
% \end{equation}
