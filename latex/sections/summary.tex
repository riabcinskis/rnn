Riabčinskis Deividas. Rekurentinio neuroninio tinklo apmokymas panaudojant grafinį vaizdo procesorių ir tinklo pritaikymas labiausiai tikėtinų žodžių pasiūlymui duotai sakinio pradžiai. Bakalauro baigiamasis projektas / vadovas Mindaugas Bražėnas; Kauno technologijos universitetas, Matematikos ir gamtos mokslų fakultetas.

Studijų kryptis ir sritis (studijų krypčių grupė): pagrindinų studijų - Taikomoji matematika (gamtos mokslai) ir gretutinių studijų Informatika (fiziniai mokslai), %(grupė mokslų nurodyti vėliau)

Kaunas 2019. %psl sk p.50

\begin{center}
\textbf{Santrauka}
\end{center}

Baigiamojo projekto darbo tikslas susidaryti ir išsivesti rekurentinio neuroninio tinklo vei\-ki\-mo formules ir tas formules realizuoti programiškai. Įvade aptariamas projekto aktualumas, suformuluojama problema ir iškeliami darbo uždaviniai. Literatūros analizės skyriuje nagrinėjami įvairūs neuroniai tinklai ir jų realizavimo metodai. LSTM rekurentinio tinklo skyriuje bus aprašytas tinklo veikimo principas, sudarytos ir išvestos formulės reikalingos tinklui apmokinti ir prognozuoti reikšmes.

\clearpage

Riabčinskis Deividas. Recurrent neural network training using GPU and its application for suggesting the most probable words given the beginning of sentence. Bachelor's Final Degree Project / Mindaugas Bražėnas; The Faculty of Mathematics and Natural Sciences, Kaunas University of Technology.


Kaunas, 2019. 48 p.

\begin{center}
\textbf{Summary}
\end{center}


\clearpage



% . Galios ir energijos matavimo mazgas lokaliam LED šviestuvui. Bakalauro baigiamasis projektas / vadovas prof. ; Kauno technologijos universitetas, Elektros ir elektronikos fakultetas.
% Studijų kryptis ir sritis (studijų krypčių grupė): Elektronikos ir elektros inžinerija, Technologijos mokslai (inžinerija).
% Reikšminiai žodžiai: galia, energija, matavimai, mikrovaldiklis.
% Kaunas 2019. 48 p.
% Santrauka
% Baigiamojo projekto darbo tikslas yra sukurti aparatinį ir programinį sprendimą LED šviestuvo suvartojamos galios ir energijos matavimui. Įvade aptariamas projekto aktualumas, suformuluojama problema ir iškeliami darbo uždaviniai.
% Problemos analizės skyriuje nagrinėjami iššūkiai, su kuriais susiduriama atliekant srovės ar įtampos matavimus, taip pat analizuojama srovės ir įtampos matavimų metodika siekiant pasirinkti tinkamiausią ir kokybiškiausią kainos atžvilgiu matavimo variantą.
% Tyrimų dalyje sudaromos matavimų grandžių simuliacijos, ištiriama komponentų nominalų bei jų tolerancijų reikšmė atliekant tikslumo reikalaujančius matavimus.
% Projektinės dalies skyriuje apžvelgiamos ir išanalizuojamos elektrinės schemos, suprojektuotos galios ir energijos matavimo prietaisui, atliekama papildomo tyrimo analizė suprojektuotam prietaisui, nuodugniai išnagrinėjami C programavimo kalbos pagrindu sudaryti programos algoritmai, taip pat pateikiami atliktų galios bei energijos matavimų tyrimų rezultatai, jų duomenų palyginimas su etaloniniu prietaisu.
%
% . Solution for power and energy measurement for local LED lamp. Bachelor's Final Degree Project / prof. ; Electrical and Electronics Faculty, Kaunas University of Technology.
% Study field and area (study field group): Electronics and Electrical Engineering, Technological sciences (Engineering).
% Keywords: power, energy, measurements, microcontroller.
% Kaunas, 2019. 48 p.
% Summary
% The aim of the final project work is to create a hardware and software solution for measuring the power and energy consumption of the LED luminaire. The introduction discusses the relevance of the project, formulates the problem and raises the tasks.
% The problem analysis section examines the challenges facing current or voltage measurements, analyzes the current and voltage measurement methodology to select the most appropriate and the best quality-price ratio measurement option.
% In the part of the researches the simulations of the measurement chains are made, the value of the component denominations and their tolerances in the measurements requiring accuracy are investigated.
% The project section reviews and analyzes the electrical schemes designed for the power and energy measuring device, the analysis of the additional research for the designed device, the program algorithms based on the C programming language are thoroughly analyzed, the results of the performed power and energy measurements as well as their data comparison with the reference device are presented.
