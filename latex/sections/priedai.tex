\subsection{ann.h failas}

//
// Random
//
class Random {
  private:
    std::mt19937 *mGen;
    std::uniform\_real\_distribution<double> *mDist;

  public:
    Random();
    double next();
    int nextInt(int min, int max);
    bool nextBool();
};


class Topology {
	private:
		std::vector<int> *ml;
	public:
		Topology();
		~Topology();

		void addLayer(int size);

		int getLayerCount();
		int getLayerSize(int index);
		int obtainNeuronCount();
		int obtainWeightCount();
		int getInputNeuronCount();
		int getOutputNeuronCount();
    void printTopology(FILE *file);
    void readTopology(FILE *file);
};

struct Derivatives{
  double *v;
  double *vh;
};

template <typename T>
class AnnBase {
  public:
    virtual void feedForward(T *h\_input, T *a, T *b) = 0; // MB: h, a->x, b->a
    virtual void backPropagation(Derivatives **deriv\_in, Derivatives **deriv\_out);
    virtual void destroy() = 0;

  private:
    virtual void prepare(Topology **top, int M)=0; // MB: M
    virtual	void init(FILE *pFile)=0; // MB: pFile neraikia kol kas
    virtual void reset()=0;
    virtual void calc\_feedForward()=0;
    virtual void copyOutput(double *a)=0;
};



class AnnSerial{ // -> AnnSerial
private:
  Topology* cTopology;
  int V; // total number of ANNs
  int u; // index of ANN
  int L;
  int M;
  int * l; // neuronu skaičius sluoksnyje
  int **vl;
  int *vL;
  int * s; // sluoksnio pradzios indeksai
  double * a\_arr;
  double * ah\_arr;
  double * z;
  int * sW;
  int ** vsW;

  double * W;
  double * dW;
  double * Wh;
  double * dWh;


  int * nG;
  int * sG;//pazieti kurie int kurie double
  double * G;

  double (*f)(double);
  double (*f\_deriv)(double);


public:

  AnnSerial(int V, int u, int M, Topology **top,  double (*f)(double), double (*f\_deriv)(double));

  void destroy();

  void feedForward(double *h\_input, double *a, double *b); // ...
  void backPropagation(Derivatives **deriv\_in, Derivatives **deriv\_out);

  void updateWeights(ErrorDerivatives *errDeriv, double alpha, double eta);

private:

  void prepare(Topology **top);
  void init(Topology **top,FILE *pFile);

  void calc\_feedForward();
  void copyOutput(double *a);

  void calcG();
  void calcDerivatives(int v, Derivatives *deriv\_h, Derivatives *deriv\_a);
  int obtainGCount(int L);
  int layerToGIndex(int L, int l);
  double d(int i, int j);

public:
  int vi(int v, int s, int i, int j, int k);
  int vhi(int v, int i, int j, int k);

  int vi(int v, int s, int i, int j);
  int vhi(int v, int i, int j);


  void setWeights(double *W, double *Wh);

  double* getWeights();
  double* getHWeights();
  double* getDWeights();
  double* getDHWeights();
  double* getA();
  double getOutput(int k);

  Topology* getTopology();
};

//
// Global functions
//

void obtainSW(Topology *top, int *sW);


\subsection{ann.cpp}


//
// Random
//
Random::Random(){
  mGen = new std::mt19937();
  mDist = new std::uniform\_real\_distribution<double>(0., 1.);
}

double Random::next(){
  return (*mDist)(*mGen);
}

int Random::nextInt(int min, int max){
  double range = max - min;
  double r = range * next();
  return min + (int)(r+0.5);
}

bool Random::nextBool(){
  if(next() >= 0) return true;
  return false;
}

//*****************************
//
// Topology
//
Topology::Topology(){
	ml = new vector<int>();
}

Topology::~Topology(){
	ml->clear();
	delete ml;
}

void Topology::addLayer(int size){
	ml->push\_back(size);
}

int Topology::getLayerCount(){
	return ml->size();
}

int Topology::getLayerSize(int index){
	return (*ml)[index];
}

int Topology::obtainNeuronCount(){
	int count = 0;
	for (int i = 0; i < ml->size(); i++)
		count += (*ml)[i] + 1;
  count--;
	return count;
}

int Topology::obtainWeightCount(){
	int count = 0;
	for (int i = 0; i < ml->size()-1; i++)
		count += ((*ml)[i] + 1)*(*ml)[i+1];
	return count;
}

int Topology::getInputNeuronCount(){
	return (*ml)[0];
}

int Topology::getOutputNeuronCount(){
	return (*ml)[ml->size()-1];
}

void Topology::printTopology(FILE *file){
  int a=getLayerCount();
  fwrite (\&a , sizeof(int), 1, file);
  for(int i=0;i<a;i++){
    int b=ml->at(i);
    fwrite (\&b , sizeof(int), 1, file);
  }
}

void Topology::readTopology(FILE *file){
  int size=0;
  (void)fread (\&size , sizeof(int), 1, file);
  int* abc=new int[size];
  ml=new vector<int>();
  (void)fread (abc , sizeof(int), size, file);
  for(int i=0;i<size;i++){
    ml->push\_back(abc[i]);
  }
}



//***********************************
//
// AnnSerial
//

AnnSerial::AnnSerial(int V, int u, int M, Topology **top, double (*f)(double), double (*f\_deriv)(double)){
  this->V = V;
  this->u = u;
  this->M = M;
  cTopology = new Topology();
    cTopology = top[u];

  L = cTopology->getLayerCount();

  this->f=(*f);
  this->f\_deriv=(*f\_deriv);

  assert(M == top[u]->getOutputNeuronCount());
  prepare(top);

  init(top, NULL);
}

void AnnSerial::destroy(){
	delete[] l;
	l = NULL;

  delete [] vl;
  vl = NULL;

  delete[] vL;
  vL= NULL;

	delete[] s;
	s = NULL;

	delete[] a\_arr;
	a\_arr = NULL;

  delete[] ah\_arr;
  ah\_arr = NULL;

	delete[] z;
	z = NULL;

	delete[] sW;
	sW = NULL;

  delete [] vsW;
  vsW = NULL;

	delete[] W;
	W = NULL;
	delete[] dW;
	dW = NULL;

  delete[] Wh;
  Wh = NULL;
  delete[] dWh;
  dWh = NULL;

  delete[] nG;
  nG = NULL;
  delete[] sG;
  sG = NULL;


	// delete[] t\_arr;
	// t\_arr = NULL;

	// delete[] gjl;
	// gjl = NULL;
}

void AnnSerial::prepare(Topology **top){

  vL = new int[V];
  vl = new int*[V];
  for(int i = 0; i < V; i++){
    vl[i] = new int[top[u]->getLayerCount()];
  }
	l = new int[cTopology->getLayerCount()];
	s = new int[cTopology->getLayerCount()];

	int neuronCount = cTopology->obtainNeuronCount();
	int weightCount = cTopology->obtainWeightCount();

	a\_arr = new double[neuronCount];
	z = new double[neuronCount];

  ah\_arr = new double[M];

  vsW = new int*[V];
  for(int v = 0; v < V; v++)
    vsW[v] = new int[top[v]->getLayerCount()-1];

	sW = new int[cTopology->getLayerCount()-1];

	W = new double[weightCount];
	dW = new double[weightCount];

  int second\_layer\_size = cTopology->getLayerSize(1);
  int h\_weightCount = second\_layer\_size * M;

  Wh = new double[h\_weightCount];
  dWh = new double[h\_weightCount];

  int G\_count = obtainGCount(L);


  nG = new int[G\_count];
  sG = new int[G\_count];

  int count = 0;
  int l2;
  for(int gl = 0; gl < G\_count; gl++){
    l2 = L - 2*gl - 1;
    nG[gl] = cTopology->getLayerSize(l2)*(cTopology->getLayerSize(l2 - 2));
    count += nG[gl];
  }


  for(int gl = 0; gl < G\_count; gl++){
    if(gl == 0)
      sG[0] = 0;
    else
      sG[gl] = sG[gl-1] + nG[gl-1];
  }

  G = new double[count];


	//gjl = new double[neuronCount];
}

void AnnSerial::init(Topology **top,FILE * pFile=NULL){

	Random *rnd = new Random();

	//Neuronu kiekiai sluoksnyje
	for (int i = 0; i < L; i++) {
		l[i] = cTopology->getLayerSize(i) + 1;
	}

	//Sluoksniu pradzios indeksai
	for (int i = 0; i < L; i++) {
		s[i] = 0;
		for (int j = i; j > 0; j--) {
			s[i] += l[j - 1];
		}
	}

	//Bias neuronai
	for (int i = 0; i < L - 1; i++) {
		a\_arr[s[i + 1] - 1] = 1;
	}

  for(int i = 0; i < V; i++){
    vL[i] = top[i]->getLayerCount();
  }

  for(int i = 0; i < V; i++)
    for(int j = 0; j < vL[i]; j++)
      vl[i][j] = top[i]->getLayerSize(j)+1;


	//Svoriu kiekiai l-ame sluoksnyje


  obtainSW(top[u], sW);
  for(int v = 0; v < V; v++){
    obtainSW(top[v], vsW[v]);
  }


  if (pFile==NULL) {
    for (int ll = 0; ll < L - 1; ll++)
      for (int i = 0; i < l[ll]; i++)
        for (int j = 0; j < l[ll+1]-1; j++){
          W[sW[ll] + i*(l[ll+1]-1) + j] = (rnd->next()*2-1);
          dW[sW[ll] + i*(l[ll+1]-1) + j] = 0.0;
        }


    int second\_layer\_size = cTopology->getLayerSize(1);
    int h\_weightCount = second\_layer\_size * M;
    for(int i = 0; i < M; i++)
      for(int j = 0; j < l[1]-1; j++)
        Wh[i*(l[1]-1) + j] = (rnd->next()*2-1);

  }
  else {
  //  readf\_Network(pFile);
  }

  delete rnd;

}

void AnnSerial::updateWeights(ErrorDerivatives *errDeriv, double alpha, double eta){

  for(int k = 0; k < cTopology->obtainWeightCount(); k++){
    dW[k] = -eta*errDeriv->v[k] + alpha*dW[k];
    W[k] = W[k] + dW[k];
  }

  for(int k = 0; k < cTopology->getLayerSize(1)*M; k++){
    dWh[k] = -eta*errDeriv->vh[k] + alpha*dWh[k];
    Wh[k] = Wh[k] + dWh[k];
  }

}

void AnnSerial::feedForward(double *h\_input, double *a, double *b){

  for (int i = 0; i < cTopology->getLayerSize(0); i++) {
    a\_arr[i] = a[i];
  }
  a\_arr[cTopology->getLayerSize(0)] = 1.0;

  for(int i=0; i<M;i++){
    ah\_arr[i] = h\_input[i];
  }

	calc\_feedForward();

	copyOutput(b);
}

void AnnSerial::calc\_feedForward(){

  for (int j = 0; j < cTopology->obtainNeuronCount(); j++) {
		z[j] = 0;
	}
  for(int k = 0; k < l[1] - 1; k++){
    for(int j = 0; j < M; j++){
      z[s[1] + k] += ah\_arr[j] * Wh[j*(l[1]-1) + k];
    }
  }
	for (int i = 0; i < L - 1; i++) {//per sluoksnius einu+
    for (int k = 0; k < l[i + 1] - 1; k++) {//per sekancio sluoksnio z+
		  for (int j = 0; j < l[i]; j++) { //kiek neuronu sluoksnyje+
				z[s[i + 1] + k] += W[sW[i] + j*(l[i + 1] - 1) + k] * a\_arr[s[i] + j];
			}
		}
		for (int k = 0; k < l[i + 1] - 1; k++) {//per sekancio sluoksnio z
			a\_arr[s[i + 1] + k] = f(z[s[i + 1] + k]);
		}
	}
}

void AnnSerial::copyOutput(double *a){
  for (int i = 0; i<cTopology->getLayerSize(cTopology->getLayerCount() - 1); i++)
		a[i] = a\_arr[s[L - 1] + i];
}

void AnnSerial::backPropagation(Derivatives **deriv\_in, Derivatives **deriv\_out){

  calcG();

  for(int v = 0; v < V; v++)
    calcDerivatives(v, deriv\_in[v],  deriv\_out[v]);
}

void AnnSerial::calcG(){
  int G\_count = obtainGCount(L);
  int l2;
  for(int gl = 0; gl < G\_count; gl++){
    l2 = L - 2*gl - 1;

    double sum = 0;

    if(l2==2){

      for(int p = 0; p < M; p++){
        for(int k = 0; k < l[l2]-1; k++){

          for(int n = 0; n < l[l2-1]-1; n++){
            sum += Wh[(l[l2-1]-1)*p + n]*W[sW[l2-1] + (l[l2]-1)*n + k]*f\_deriv(z[s[l2-1]+n]);
            //pakeiciau W i Wh
            //sum += Wh[M*p + n]*W[sW[l2-1] + (l[l2]-1)*n + k]*f\_deriv(z[s[l2-1]+n]);
          }

          G[sG[gl] + (l[l2]-1)*p + k] = f\_deriv(z[s[l2] + k])*sum;
        }

      //  G[sG[gl] + (l[l2]-1)*p + k] = f\_deriv(z[s[l2] + k])*sum;
      }

    }else{
      for(int p = 0; p < l[l2-2]; p++){
        for(int k = 0; k < l[l2]-1; k++){

          for(int n = 0; n < l[l2-1]-1; n++){
            sum += W[sW[l2-2] + (l[l2-1]-1)*p + n]*W[sW[l2-1] + (l[l2]-1)*n + k]*f\_deriv(z[s[l2-1]+n]);
          }

          G[sG[gl] + (l[l2]-1)*p + k] = f\_deriv(z[s[l2] + k])*sum;
        }
      }
    }
  }
}

void AnnSerial::calcDerivatives(int v, Derivatives *deriv\_in, Derivatives *deriv\_out){
  // deriv\_in->v[vi(...)];
  // deriv\_in->vh[vhi(...)];

  int max\_layer\_size = 0;
  for(int i = 0; i < L; i++)
    max\_layer\_size = (l[i]-1) > max\_layer\_size ? l[i]-1 : max\_layer\_size;

  // vec0 -> vec1
  double *vec0 = new double[max\_layer\_size];
  double *vec1 = new double[max\_layer\_size];


  int start\_l = L%2 == 1 ? 0 : 1;

    for(int ss = 0; ss < vL[v]-1; ss++){//kiek sluoksniu tinkle
      for(int wi = 0; wi < vl[v][ss]; wi++){//kiek neuronu
        for(int wj = 0; wj < vl[v][ss+1]-1; wj++){//kiek sekanciame neuronu
          for(int ll = start\_l; ll < L; ll=ll+2){//per G reiksmes  cia gal maziau tik

            if(ll == 0){

              for(int k = 0; k < M; k++)
                vec1[k] = deriv\_in->v[vi(v, ss, wi, wj, k)];

            }else if(ll == 1){

              for(int k = 0; k < l[ll]-1;k++){
                double sum = 0;
                for(int n = 0; n < M; n++){
                  sum += Wh[(l[ll]-1)*n + k]*deriv\_in->v[vi(v, ss, wi, wj, n)];
                }

                if(u == v) sum += a\_arr[wi];

                vec1[k] = f\_deriv(z[s[ll] + k])*sum;
              }


            }else {


              int gl = layerToGIndex(L, ll);
              double sum = 0;
              int to\_p = ll == 2 ? M : l[ll-2]-1;//-1 gal


              for(int k = 0; k < l[ll]-1;k++){
                for(int p = 0; p < to\_p; p++){
                  sum += vec0[p]*G[sG[gl] + (l[ll]-1)*p + k];
                }


                if(u == v){
                  if(ll == ss+1)
                    sum += f\_deriv(z[s[ll] + k])*a\_arr[s[ll-1] + wi];
                  else if(ll == ss+2){
                      double sum2 = 0;
                      for(int n = 0; n < l[ll-1]; n++)
                       sum2 += W[sW[ll-1] + (l[ll]-1)*n + k] * f\_deriv(z[s[ll-1] + n]);
                      sum += f\_deriv(z[s[ll] + k])*a\_arr[s[ll-2] + wi]*sum2;
                  }
                }
              }
            }
            int to\_k = ll == 0 ? M : l[ll]-1;
            for(int k = 0; k < to\_k; k++) vec0[k] = vec1[k];

          }
          for(int k = 0; k < M /*l[L-1]-1*/; k++){
            deriv\_out->v[vi(v, ss, wi, wj, k)] = vec1[k];

          }
        }
      }
    }

      for(int wi = 0; wi < M; wi++){//kiek neuronu
        for(int wj = 0; wj < vl[v][1]-1; wj++){//kiek sekanciame neuronu
          for(int ll = start\_l; ll < L; ll=ll+2){//per G reiksmes  cia gal maziau tik
            if(ll == 0){

              for(int k = 0; k < M; k++)
                vec1[k] = deriv\_in->vh[vhi(v, wi, wj, k)];

            }else if(ll == 1){

              for(int k = 0; k < l[ll]-1;k++){
                double sum = 0;
                for(int n = 0; n < M; n++)
                  sum += Wh[(l[ll]-1)*n + k]*deriv\_in->vh[vhi(v, wi, wj, n)];
                if(u == v ) sum += ah\_arr[wi];
                vec1[k] = f\_deriv(z[s[ll] + k])*sum;

              }

            }else {


              int gl = layerToGIndex(L, ll);
              double sum = 0;
              int to\_p = ll == 2 ? M : l[ll-2]-1;//-1 gal


              for(int k = 0; k < l[ll]-1;k++){
                for(int p = 0; p < to\_p; p++){
                  sum += vec0[p]*G[sG[gl] + (l[ll]-1)*p + k];
                }


                if(u == v){

                  if(ll == 2){
                    double sum2 = 0;
                    for(int n = 0; n < l[ll-1]; n++)
                      sum2 += W[sW[ll-1] + (l[ll]-1)*n + k] * f\_deriv(z[s[ll-1] + n]);
                    sum += f\_deriv(z[s[ll] + k])*ah\_arr[wi]*sum2;
                  }
                }
              }
            }

            int to\_k = ll == 0 ? M : l[ll]-1;
            for(int k = 0; k < to\_k; k++) vec0[k] = vec1[k];
          }

          for(int k = 0; k < M /*l[L-1]-1*/; k++){
            deriv\_out->vh[vhi(v, wi, wj, k)] = vec1[k];
          }
        }
      }

  delete [] vec0;
  delete [] vec1;

}

int AnnSerial::obtainGCount(int L){
  int G\_count = 0;
  int remaining\_L = L;
  while(remaining\_L >= 3){
    G\_count++;
    remaining\_L -= 2;
  }
  return G\_count;
}

int AnnSerial::layerToGIndex(int L, int l){
  int G\_count = obtainGCount(L);
  return G\_count - (l - l%2) / 2;
}

int AnnSerial::vi(int v, int s, int i, int j, int k){
  return (vsW[v][s] + i*(vl[v][s+1]-1) + j)*M + k;
}

int AnnSerial::vhi(int v, int i, int j, int k){
  return  (i*(vl[v][1]-1) + j)*M + k;
}

int AnnSerial::vi(int v, int s, int i, int j){
  return vsW[v][s] + i*(vl[v][s+1]-1) + j;
}

int AnnSerial::vhi(int v, int i, int j){
  return  i*(vl[v][1]-1) + j;
}

double AnnSerial::d(int i, int j){
  if(i == j) return 1.0;
  return 0.0;
}

void AnnSerial::setWeights(double *W, double *Wh){
  this->W = W;
  this->Wh = Wh;
};

double* AnnSerial::getWeights(){
	return W;
}

double* AnnSerial::getDWeights(){
	return dW;
}

double* AnnSerial::getHWeights(){
	return Wh;
}

double* AnnSerial::getDHWeights(){
	return dWh;
}

double* AnnSerial::getA(){
	return a\_arr;
}

double AnnSerial::getOutput(int k){
  return a\_arr[s[L-1]+k];
}


Topology* AnnSerial::getTopology(){
  return cTopology;
}

//
// Global functions
//


void obtainSW(Topology *top, int *sW){
  int nW;
  sW[0] = 0;
  for (int i = 1; i < top->getLayerCount()-1; i++) {
		nW = (top->getLayerSize(i-1)+1)*top->getLayerSize(i); //l[i] * (l[i + 1] - 1);
    sW[i] = sW[i-1] + nW;
  }
}

\subsection{rnn.h}

class Topology;
class AnnSerial;
struct Derivatives;

/* Class definitions here. */
void run\_cuda\_sample();

class RnnConfig{
private:
  Topology** cTopology;
  int mM;


public:
  void setTopologies(Topology **top);
  void setM(int M);


  Topology** getTopologies();
  int getM();

};



struct RnnDerivatives {
  Derivatives **hderiv;
  //+hderiv : Derivatives[]
  Derivatives **cderiv;
  //+cderiv : Derivatives[]
};

struct ErrorDerivatives {
  double *v;
  //+v : double[]
  double *vh;
  //+vh : double[]
};

template <typename T>
class RnnBase {
  public:
  	 virtual void train(T *a, T *b, T alpha, T eta) = 0;
  	 virtual void feedForward(T *h\_in, T *c\_in,T *a, T *c\_out, T *h\_out) = 0;


  private:
     virtual	void init(FILE *pFile)=0;
};


class RnnCell {
  private:

    int M;
    //-M : int
    int V;
    //-V : int

    AnnSerial** anns;
    //-anns : AnnSerial[]
    AnnSerial* ann\_forget; // 0

    AnnSerial* ann\_input; // 1
    AnnSerial* ann\_gate; // 2
    AnnSerial* ann\_output; // 3

    double* c\_current;
    //-c\_current : double[]
    double* c\_new;
    //-c\_new : double[]
    double* h\_current;
    //-h\_current : double[]
    double* h\_new;
    //-h\_new : double[]
    double* b;
    //-b : double[]
    double** a\_outputs;
    //-a\_outputs : double[][]

    Derivatives ***aderiv;
    //-aderiv : Derivatives[][]




  public:

    RnnCell(int M, Topology **top) {
      prepare(M, top);
      init(NULL);
    };
    //+RnnCell(M : int, top : Topology[])


    RnnCell(int M, string filename);



  	 void feedForward(double *h\_in, double *c\_in,double *a, double *c\_out, double *h\_out);
     //+feedForward(h\_in : double[], c\_in : double[],a : double[], c\_out : double[], h\_out : double[]) : void
     void backPropagation( RnnDerivatives *deriv\_in, RnnDerivatives *deriv\_out);
     //+backPropagation( deriv\_in : RnnDerivatives, deriv\_out : RnnDerivatives) : void

    //
    AnnSerial* getANN(int v);
    //+getANN(v : int) : AnnSerial

    void destroy();
    //+destroy() : void

    void printf\_Network(string output\_filename);

    //+printf\_Network(output\_filename : string) : void
  private:
    void prepare(int M, Topology **top);
    //-prepare(M : int, top : Topology[]) : void
    void init(FILE *pFile);
    //-init(pFile : FILE) : void

};


class OutputLimit {
  public:
    virtual void reset() = 0;
    virtual bool check(double *vec) = 0;
};

class SecondMarkLimit : public OutputLimit {
  private:
    int M;
    int count;
    int markIndex;

  public:
    SecondMarkLimit(int markIndex, int M);

    void reset();
    bool check(double *vec);
};

struct DataNode{
  DataNode(int M);
  double* vec;
  DataNode *next;
};


class Rnn {
  private:
    int I;
    int M;
    int V;

    RnnCell* cRnnCell;


    double *h\_in;
    double *h\_out;
    double* c\_in;
    double *c\_out;

    ErrorDerivatives** errDeriv;
    RnnDerivatives **rnnDeriv;

    int impl;

  public:
    Rnn(int I, int M, RnnCell *rnnCell, int impl);
    Rnn(int I, int M, RnnCell *rnnCell);

    DataNode* feedForward(DataNode* input, OutputLimit *outputLimit);
    bool backPropagation(DataNode* input, DataNode* output, OutputLimit *outputLimit, double \&error);

    void updateWeights(double alpha, double eta);
    //+updateWeights(alpha : double, eta : double) : void
    void resetErrorDerivatives();
    //+resetErrorDerivatives() : void


    RnnCell* getRnnCell(){
      return cRnnCell;
    }

    //+getRnnCell() : RnnCell

  private:

    RnnDerivatives* allocateRnnDerivatives(RnnDerivatives* deriv);
    //-allocateRnnDerivatives(deriv : RnnDerivatives) : RnnDerivatives

    //RnnDerivatives* deallocateRnnDerivatives(RnnDerivatives* deriv);
    void resetHDerivatives(Derivatives** hderiv);
    //-resetHDerivatives(hderiv : Derivatives[]) : void
    void resetCDerivatives(Derivatives** cderiv);


    void copyRnnDerivatives(RnnDerivatives* deri\_b, RnnDerivatives* deriv\_a);
    //-copyRnnDerivatives(deri\_b : RnnDerivatives, deriv\_a : RnnDerivatives) : void
    void copyVector(double* vec\_b, double *vec\_a, int n);
    //-copyVector(vec\_b : double[], vec\_a : double[], n : int) : void

    void sumErrorDerivatives(double *h, Derivatives **hderiv, double *y);
    //-sumErrorDerivatives(h : double, hderiv : Derivatives[], y : double[]) : void
    double calcError(double *h, double *y);
    //-calcError(h : double[], y : double[]) : double


};

double f(double x);
double f\_deriv(double x);

double f\_tanh(double x);
double f\_tanh\_deriv(double x);



\subsection{rnn.cpp}

/* C++ code here. */
void RnnConfig::setTopologies(Topology **top){
  cTopology = top;
}

void RnnConfig::setM(int M){
  mM=M;
}


Topology** RnnConfig::getTopologies(){
  return cTopology;
}

int RnnConfig::getM(){
  return mM;
}


//
// RnnCell
//

RnnCell::RnnCell(int M, string filename) {
  FILE * p1File;
  p1File = fopen(filename.c\_str(), "rb");

  Topology** top= new Topology*[4];
  top[0] = new Topology();
  top[1] = new Topology();
  top[2] = new Topology();
  top[3] = new Topology();
  top[0]->readTopology(p1File);
  top[1]->readTopology(p1File);
  top[2]->readTopology(p1File);
  top[3]->readTopology(p1File);


  prepare(M,top);

  double* Ww0= new double[top[0]->obtainWeightCount()];
  double* Ww1= new double[top[1]->obtainWeightCount()];
  double* Ww2= new double[top[2]->obtainWeightCount()];
  double* Ww3= new double[top[3]->obtainWeightCount()];

  double* Whw0= new double[M*top[0]->getLayerSize(1)];
  double* Whw1= new double[M*top[1]->getLayerSize(1)];
  double* Whw2= new double[M*top[2]->getLayerSize(1)];
  double* Whw3= new double[M*top[3]->getLayerSize(1)];

  (void)fread (Ww0 , sizeof(double), top[0]->obtainWeightCount(), p1File);
  (void)fread (Whw0 , sizeof(double), M*top[0]->getLayerSize(1), p1File);

  (void)fread (Ww1 , sizeof(double), top[1]->obtainWeightCount(), p1File);
  (void)fread (Whw1 , sizeof(double), M*top[1]->getLayerSize(1), p1File);

  (void)fread (Ww2 , sizeof(double), top[2]->obtainWeightCount(), p1File);
  (void)fread (Whw2 , sizeof(double), M*top[2]->getLayerSize(1), p1File);

  (void)fread (Ww3 , sizeof(double), top[3]->obtainWeightCount(), p1File);
  (void)fread (Whw3 , sizeof(double), M*top[3]->getLayerSize(1), p1File);


  anns[0]->setWeights(Ww0,Whw0);

  anns[1]->setWeights(Ww1,Whw1);

  anns[2]->setWeights(Ww2,Whw2);

  anns[3]->setWeights(Ww3,Whw3);



  init(p1File);
  fclose (p1File);
};


void RnnCell::printf\_Network(string output\_filename){

    FILE * pFile;
    const char * c = output\_filename.c\_str();
    pFile = fopen(c, "wb");
    anns[0]->getTopology()->printTopology(pFile);
    anns[1]->getTopology()->printTopology(pFile);
    anns[2]->getTopology()->printTopology(pFile);
    anns[3]->getTopology()->printTopology(pFile);

    fwrite (anns[0]->getWeights() , sizeof(double), anns[0]->getTopology()->obtainWeightCount(), pFile);
    fwrite (anns[0]->getHWeights() , sizeof(double), anns[0]->getTopology()->getLayerSize(1)*M, pFile);

    fwrite (anns[1]->getWeights() , sizeof(double), anns[1]->getTopology()->obtainWeightCount(), pFile);
    fwrite (anns[1]->getHWeights() , sizeof(double), anns[1]->getTopology()->getLayerSize(1)*M, pFile);

    fwrite (anns[2]->getWeights() , sizeof(double), anns[2]->getTopology()->obtainWeightCount(), pFile);
    fwrite (anns[2]->getHWeights() , sizeof(double), anns[2]->getTopology()->getLayerSize(1)*M, pFile);

    fwrite (anns[3]->getWeights() , sizeof(double), anns[3]->getTopology()->obtainWeightCount(), pFile);
    fwrite (anns[3]->getHWeights() , sizeof(double), anns[3]->getTopology()->getLayerSize(1)*M, pFile);

    fclose (pFile);
}

void RnnCell::prepare(int M, Topology **top){
  this->M = M;
  V = 4;

  double (*func)(double);
  double (*func\_deriv)(double);

  double (*func\_tanh)(double);
  double (*func\_tanh\_deriv)(double);

  func=f;
  func\_deriv = f\_deriv;

  func\_tanh=f\_tanh;
  func\_tanh\_deriv = f\_tanh\_deriv;


  anns = new AnnSerial*[V];
  for(int v = 0; v < V; v++){
    if(v==2){
      anns[v] = new AnnSerial(V, v, M, top, func\_tanh, func\_tanh\_deriv);
    } else {
      anns[v] = new AnnSerial(V, v, M, top, func, func\_deriv);
    }
  }

  ann\_forget = anns[0];
  ann\_input = anns[1];
  ann\_gate = anns[2];
  ann\_output = anns[3];




  b = new double[M];
  c\_current = new double[M];
  c\_new = new double[M];
  h\_current = new double[M];
  h\_new = new double[M];

  a\_outputs = new double*[V];
  for(int v = 0; v < V; v++){
    a\_outputs[v] = new double[M];
  }

  aderiv = new Derivatives**[V];
  for(int u = 0; u < V; u++){
    aderiv[u] = new Derivatives*[V];
    for(int v = 0; v < V; v++){
      aderiv[u][v] = new Derivatives;
      aderiv[u][v]->v = new double[top[v]->obtainWeightCount()*M];
      aderiv[u][v]->vh = new double[M*top[v]->getLayerSize(1)*M];
    }
  }
}


void RnnCell::init(FILE * pFile=NULL){

  for(int i=0; i<M; i++){
    c\_current[i] = 0;
    h\_current[i] = 0;
  }
}

void RnnCell::feedForward(double *h\_in, double *c\_in, double *a\_in, double *c\_out, double *h\_out){

  c\_current = c\_in; // c\_t0, c\_t1,
  h\_current = h\_in;

  for(int v = 0; v < V; v++)
    anns[v]->feedForward(h\_in, a\_in, a\_outputs[v]);

  for(int i=0; i < M; i++){
    c\_new[i] = c\_out[i] = c\_in[i] * a\_outputs[0][i] + a\_outputs[1][i] * a\_outputs[2][i];
    b[i] = tanh(c\_out[i])* a\_outputs[3][i];
  }

  double sumB = 0;
  for(int i = 0; i < M; i++)
    sumB += exp(b[i]);

  for(int i = 0; i < M; i++)
    h\_new[i] = h\_out[i] = exp(b[i]) / sumB;

}


void RnnCell::backPropagation(RnnDerivatives *deriv\_in, RnnDerivatives *deriv\_out){



  for(int u = 0; u < V; u++)
    anns[u]->backPropagation(deriv\_in->hderiv, aderiv[u]);



  for(int v = 0; v < V; v++){

    Topology *vtop = anns[v]->getTopology();
    for(int s = 0; s < vtop->getLayerCount()-1; s++){

      for(int wi = 0; wi < vtop->getLayerSize(s)+1; wi++){

        for(int wj = 0; wj < vtop->getLayerSize(s+1); wj++){


          for(int k = 0; k < M; k++){

            double cderiv = deriv\_in->cderiv[v]->v[anns[v]->vi(v, s, wi, wj, k)];
            // printf("cderiv: %.20f\n", cderiv);
            double a\_forget = ann\_forget->getOutput(k);
            // printf("a\_forget: %.20f\n", a\_forget);
            double c = c\_current[k];
            // printf("c: %.20f\n", c);
            double a\_forget\_deriv = aderiv[0][v]->v[anns[v]->vi(v, s, wi, wj, k)];
            // printf("a\_forget\_deriv: %.20f\n", a\_forget\_deriv);
            double a\_input\_deriv = aderiv[1][v]->v[anns[v]->vi(v, s, wi, wj, k)];
            // printf("a\_input\_deriv: %.20f\n", a\_input\_deriv);
            double a\_gate = ann\_gate->getOutput(k);
            // printf("a\_gate: %.20f\n", a\_gate);
            double a\_input = ann\_input->getOutput(k);
            // printf("a\_input: %.20f\n", a\_input);
            double a\_gate\_deriv = aderiv[2][v]->v[anns[v]->vi(v, s, wi, wj, k)];
            // printf("a\_gate\_deriv: %.20f\n", a\_gate\_deriv);

            deriv\_out->cderiv[v]->v[anns[v]->vi(v, s, wi, wj, k)] = cderiv*a\_forget + c*a\_forget\_deriv + a\_input\_deriv*a\_gate + a\_input*a\_gate\_deriv;
            // printf("cderiv*a\_forget:  %.20f\n", cderiv*a\_forget);
            // printf("c*a\_forget\_deriv:  %.20f\n", c*a\_forget\_deriv);
            // printf("a\_input\_deriv*a\_gate:  %.20f\n", a\_input\_deriv*a\_gate);
            // printf("a\_input*a\_gate\_deriv:  %.20f\n", a\_input*a\_gate\_deriv);
            // printf("cia:  %.20f\n", deriv\_out->cderiv[0]->v[0]);
            // printf("cia:  %.20f\n", deriv\_out->cderiv[0]->v[1]);
            // break;
          }
        }
      }
    }
    // printf("cia:  %.20f\n", deriv\_out->cderiv[0]->v[0]);
    // printf("cia:  %.20f\n", deriv\_out->cderiv[0]->v[1]);


    for(int wi = 0; wi < M /*vtop->getLayerSize(0)*/; wi++){
      for(int wj = 0; wj < vtop->getLayerSize(1); wj++){

        for(int k = 0; k < M; k++){

          double cderiv = deriv\_in->cderiv[v]->vh[anns[v]->vhi(v, wi, wj, k)];
          // if(wi==1 \&\& wj==0\&\& k==0 \&\& v==0) printf("cderiv: %.20f\n", cderiv);

          double a\_forget = ann\_forget->getOutput(k);
          // if(wi==1 \&\& wj==0\&\& k==0\&\& v==0) printf("a\_forget: %.20f\n", a\_forget);

          double c = c\_current[k];
          // if(wi==1 \&\& wj==0\&\& k==0\&\& v==0) printf("c: %.20f\n", c);

          double a\_forget\_deriv = aderiv[0][v]->vh[anns[v]->vhi(v, wi, wj, k)];
          // if(wi==1 \&\& wj==0\&\& k==0\&\& v==0) printf("a\_forget\_deriv: %.20f\n", a\_forget\_deriv);

          double a\_input\_deriv = aderiv[1][v]->vh[anns[v]->vhi(v, wi, wj, k)];
          // if(wi==1 \&\& wj==0\&\& k==0\&\& v==0) printf("a\_input\_deriv: %.20f\n", a\_input\_deriv);

          double a\_gate = ann\_gate->getOutput(k);
          // if(wi==1 \&\& wj==0\&\& k==0\&\& v==0) printf("a\_gate: %.20f\n", a\_gate);

          double a\_input = ann\_input->getOutput(k);
          // if(wi==1 \&\& wj==0\&\& k==0\&\& v==0) printf("a\_input: %.20f\n", a\_input);

          double a\_gate\_deriv = aderiv[2][v]->vh[anns[v]->vhi(v, wi, wj, k)];
          // if(wi==1 \&\& wj==0\&\& k==0\&\& v==0) printf("a\_gate\_deriv: %.20f\n", a\_gate\_deriv);

          deriv\_out->cderiv[v]->vh[anns[v]->vhi(v, wi, wj, k)] = cderiv*a\_forget + c*a\_forget\_deriv + a\_input\_deriv*a\_gate + a\_input*a\_gate\_deriv;
          // if(wi==1 \&\& wj==0 \&\& k==0\&\& v==0) printf("cderiv*a\_forget:  %.20f\n", cderiv*a\_forget);
          // if(wi==1 \&\& wj==0 \&\& k==0\&\& v==0) printf("c*a\_forget\_deriv:  %.20f\n", c*a\_forget\_deriv);
          // if(wi==1 \&\& wj==0 \&\& k==0\&\& v==0) printf("a\_input\_deriv*a\_gate:  %.20f\n", a\_input\_deriv*a\_gate);
          // if(wi==1 \&\& wj==0 \&\& k==0\&\& v==0) printf("a\_input*a\_gate\_deriv:  %.20f\n", a\_input*a\_gate\_deriv);
          // if(wi==1 \&\& wj==0 \&\& k==0\&\& v==0) printf("deriv\_out->cderiv[v]->vh[anns[v]->vhi(v, wi, wj, k)]: %.20f\n", deriv\_out->cderiv[v]->vh[anns[v]->vhi(v, wi, wj, k)]);
        }
      }
    }
  }

  // printf("1%.20f\n", deriv\_out->cderiv[0]->vh[0]);
  // printf("%.20f\n", deriv\_out->cderiv[0]->vh[1]);
  // printf("%.20f\n", deriv\_out->cderiv[0]->vh[2]);
  // printf("4%.20f\n", deriv\_out->cderiv[0]->vh[3]);
  double* sm\_deriv = new double[M*M];


  for(int k = 0; k < M; k++)
    for(int n = 0; n < M; n++)
      if(k == n) sm\_deriv[k*M + n] = h\_new[k]*(1-h\_new[k]);
      else sm\_deriv[k*M + n] = -h\_new[k]*h\_new[n];

      // printf("sm\_deriv[k*M]: %.20f\n", sm\_deriv[0*M]);
      // printf("sm\_deriv[k*M]: %.20f\n", sm\_deriv[1]);

      for(int v = 0; v < V; v++){
        Topology *vtop = anns[v]->getTopology();
        for(int s = 0; s < vtop->getLayerCount()-1; s++){
          for(int wi = 0; wi < vtop->getLayerSize(s)+1; wi++){
            for(int wj = 0; wj < vtop->getLayerSize(s+1); wj++){
              for(int k = 0; k < M; k++){
                double sum = 0;
                for(int n = 0; n < M; n++){
                  double sum2=0;
                  //sm\_deriv[k*M + n]
                  double a\_output\_deriv = aderiv[3][v]->v[anns[v]->vi(v, s, wi, wj, n)];
                  // printf("a\_output\_deriv: %.20f\n", a\_output\_deriv);
                  double a\_output = ann\_output->getOutput(n);
                  // printf("a\_output: %.20f\n", a\_output);
                  double c\_deriv = deriv\_out->cderiv[v]->v[anns[v]->vi(v, s, wi, wj, n)];
                  // printf("c\_deriv: %.20f\n", c\_deriv);
                  sum2 += a\_output\_deriv*f\_tanh(c\_new[n]) + a\_output*f\_tanh\_deriv(c\_new[n])*c\_deriv;
                  // printf("a\_output\_deriv*f\_tanh(c\_new[n]): %.20f\n", a\_output\_deriv*f\_tanh(c\_new[n]));
                  // printf("a\_output*f\_tanh\_deriv(c\_new[n])*c\_deriv: %.20f\n", a\_output*f\_tanh\_deriv(c\_new[n])*c\_deriv);
                  sum2 *= sm\_deriv[k*M+n];

                  sum+=sum2;

                  // printf("sum: %.20f\n", sum);
                }
                deriv\_out->hderiv[v]->v[anns[v]->vi(v, s, wi, wj, k)] = sum;

              }
            }
          }
        }


        for(int wi = 0; wi < M /*vtop->getLayerSize(0)*/; wi++){
          for(int wj = 0; wj < vtop->getLayerSize(1); wj++){
            for(int k = 0; k < M; k++){
              double sum = 0;
              for(int n = 0; n < M; n++){
                //sm\_deriv[k*M + n]
                double a\_output\_deriv = aderiv[3][v]->vh[anns[v]->vhi(v, wi, wj, n)];
                // printf("a\_output\_deriv:%.20f\n", a\_output\_deriv);

                double a\_output = ann\_output->getOutput(n);
                // printf("a\_output:%.20f\n", a\_output);
                double c\_deriv = deriv\_out->cderiv[v]->vh[anns[v]->vhi(v, wi, wj, n)];
                // printf("c\_deriv:%.20f\n", c\_deriv);
                sum += a\_output\_deriv*f\_tanh(c\_new[n]) + a\_output*f\_tanh\_deriv(c\_new[n])*c\_deriv;
                sum *= sm\_deriv[k*M+n];
              }
              deriv\_out->hderiv[v]->vh[anns[v]->vhi(v, wi, wj, k)] = sum;

            }
          }
        }
      }

  delete [] sm\_deriv;

}

void RnnCell::destroy(){
    for(int i = 0; i < V; i++){
      anns[i]->destroy();
    }

   delete [] anns;
   anns = NULL;


  delete c\_current;
  c\_current = NULL;
  delete c\_new;
  c\_new = NULL;
  delete h\_current;
  h\_current = NULL;
  delete h\_new;
  h\_new = NULL;
  delete b;
  b = NULL;
  delete [] a\_outputs;
  a\_outputs = NULL;

  delete ann\_forget; // 0
  delete ann\_input; // 1
  delete ann\_gate; // 2
  delete ann\_output; // 3

  delete *aderiv;

}


AnnSerial* RnnCell::getANN(int v){
  return anns[v];
}


//
// SecondMarkLimit
//

SecondMarkLimit::SecondMarkLimit(int markIndex, int M){
  this->markIndex = markIndex;
  this->M = M;
}

void SecondMarkLimit::reset(){
  count = 0;
}

bool SecondMarkLimit::check(double *vec){
  int maxAt = 0;
  for(int i = 1; i < M; i++)
    maxAt = vec[maxAt] < vec[i] ? i : maxAt;

  if(maxAt == markIndex) count++;

  if(count == 2) return true;
  return false;
}


//
// DataNode
//
DataNode::DataNode(int M){
  vec = new double[M];
  next = NULL;
}

//
// Rnn
//

Rnn::Rnn(int I, int M, RnnCell *rnnCell){

  impl = RNN\_FULL\_BACKPROPAGATION;
  // printf("M:%d\n", M);
  Rnn(I, M, rnnCell, impl);
}

Rnn::Rnn(int I, int M, RnnCell *rnnCell, int impl){

  this->I = I;
  this->M = M;
  this->V = 4;
  cRnnCell = rnnCell;
  this->impl = impl;
  // printf("M:%d\n", this->M);
  h\_in = new double[M];
  h\_out = new double[M];
  c\_in = new double[M];
  c\_out = new double[M];
  // printf("%s\n", "buvo");
  rnnDeriv = new RnnDerivatives*[2];
  rnnDeriv[0] = new RnnDerivatives;
  rnnDeriv[1] = new RnnDerivatives;
  allocateRnnDerivatives(rnnDeriv[0]);
  allocateRnnDerivatives(rnnDeriv[1]);

  errDeriv = new ErrorDerivatives*[V];

  for(int v = 0; v < V; v++){
    errDeriv[v] = new ErrorDerivatives;
    errDeriv[v]->v = new double[cRnnCell->getANN(v)->getTopology()->obtainWeightCount()];
    errDeriv[v]->vh = new double[cRnnCell->getANN(v)->getTopology()->getLayerSize(1)*M];
  }
}

DataNode* Rnn::feedForward(DataNode* input, OutputLimit *outputLimit){
  // printf("%d\n", M);
  for(int k = 0; k < M; k++){
    h\_in[k] = 0;
    c\_in[k] = 0;
  }

  DataNode* p = input;
  while(p->next != NULL){


    cRnnCell->feedForward(h\_in, c\_in, p->vec, c\_out, h\_out);
    copyVector(h\_in, h\_out, M);
    copyVector(c\_in, c\_out, M);

    p = p->next;
  }

  DataNode* out = new DataNode(M);
  DataNode* q = out;

  outputLimit->reset();

  cRnnCell->feedForward(h\_in, c\_in, p->vec, c\_out, h\_out);
  copyVector(h\_in, h\_out, M);
  copyVector(c\_in, c\_out, M);
  copyVector(q->vec, h\_out, M);

  if(outputLimit->check(q->vec)) return out;

 double* empty\_input = new double[I];
 for(int i = 0; i < I; i++)
  empty\_input[i] = 0;


  int count=0;
  do{
    count++;
    // printf("%d\n", count);
    cRnnCell->feedForward(h\_in, c\_in, empty\_input, c\_out, h\_out);
    copyVector(h\_in, h\_out, M);
    copyVector(c\_in, c\_out, M);
    // printf("%s\n", "asdsa");
    q->next = new DataNode(M);
    q = q->next;
    copyVector(q->vec, h\_out, M);




  }while(outputLimit->check(q->vec) == false \&\& count < 100);

  delete [] empty\_input;

  return out;
}

bool Rnn::backPropagation(DataNode* input, DataNode* output, OutputLimit *outputLimit, double \&error){

  for(int k = 0; k < M; k++){
    h\_in[k] = 0;
    c\_in[k] = 0;
  }

  error = 0;

  outputLimit->reset();

  int derivIndex = 0;
  resetHDerivatives(rnnDeriv[derivIndex]->hderiv);
  resetCDerivatives(rnnDeriv[derivIndex]->cderiv);

  //initRnnDerivatives(rnnDeriv[1 - derivIndex]);

  DataNode* p = input;


//printf("----------------\\n");

  while(p->next != output){
    //printf("A");
    cRnnCell->feedForward(h\_in, c\_in, p->vec, c\_out, h\_out);
    if(impl == RNN\_APPROX\_BACKPROPAGATION)resetHDerivatives(rnnDeriv[derivIndex]->hderiv);
    cRnnCell->backPropagation(rnnDeriv[derivIndex], rnnDeriv[1-derivIndex]);

    copyVector(h\_in, h\_out, M);
    copyVector(c\_in, c\_out, M);
    p = p->next;
    derivIndex = 1 - derivIndex;
  }



  cRnnCell->feedForward(h\_in, c\_in, p->vec, c\_out, h\_out);
  if(impl == RNN\_APPROX\_BACKPROPAGATION)resetHDerivatives(rnnDeriv[derivIndex]->hderiv);
  cRnnCell->backPropagation(rnnDeriv[derivIndex], rnnDeriv[1-derivIndex]);
  copyVector(h\_in, h\_out, M);
  copyVector(c\_in, c\_out, M);

  sumErrorDerivatives(h\_out, rnnDeriv[1-derivIndex]->hderiv, output->vec);
  error += calcError(h\_out, output->vec);
  int outputCount = 1;


  derivIndex = 1 - derivIndex;
  DataNode* q = output;

  double* empty\_input = new double[I];
  for(int i = 0; i < I; i++)
   empty\_input[i] = 0;

   //char abc[64]=" abcdefghijklmnopqrstuvwxyz-";
   //char abc[64]=" 0123456789";

   do{

     q = q->next;
     //printf("%c", vec\_to\_char(abc, q->vec));
    // printf("B");
     if(q == NULL) {

       delete [] empty\_input;
       return false;
     }

     cRnnCell->feedForward(h\_in, c\_in, empty\_input, c\_out, h\_out);
     if(impl == RNN\_APPROX\_BACKPROPAGATION)resetHDerivatives(rnnDeriv[derivIndex]->hderiv);
     cRnnCell->backPropagation(rnnDeriv[derivIndex], rnnDeriv[1-derivIndex]);
     copyVector(h\_in, h\_out, M);
     copyVector(c\_in, c\_out, M);

     sumErrorDerivatives(h\_out, rnnDeriv[1-derivIndex]->hderiv, q->vec);
     error += calcError(h\_out, q->vec);
     outputCount++;

     derivIndex = 1 - derivIndex;

   }while(outputLimit->check(q->vec) == false);


   delete [] empty\_input;

   error = error / (double)outputCount;

   return true;

}

void Rnn::updateWeights(double alpha, double eta){
  for(int v = 0; v < V; v++)
    cRnnCell->getANN(v)->updateWeights(errDeriv[v], alpha, eta);
}

void Rnn::resetErrorDerivatives(){
  for(int v = 0; v < V; v++){
    for(int k = 0; k < cRnnCell->getANN(v)->getTopology()->obtainWeightCount(); k++)
      errDeriv[v]->v[k] = 0.0;
    for(int k = 0; k < cRnnCell->getANN(v)->getTopology()->getLayerSize(1)*M; k++)
      errDeriv[v]->vh[k] = 0.0;
  }
}


// PRIVATE


RnnDerivatives* Rnn::allocateRnnDerivatives(RnnDerivatives* deriv){
  deriv->hderiv = new Derivatives*[V];
  deriv->cderiv = new Derivatives*[V];

  for(int v = 0; v < V; v++){
    deriv->hderiv[v] = new Derivatives;
    deriv->hderiv[v]->v = new double[cRnnCell->getANN(v)->getTopology()->obtainWeightCount()*M];
    deriv->hderiv[v]->vh = new double[cRnnCell->getANN(v)->getTopology()->getLayerSize(1)*M*M];

    deriv->cderiv[v] = new Derivatives;
    deriv->cderiv[v]->v = new double[cRnnCell->getANN(v)->getTopology()->obtainWeightCount()*M];
    deriv->cderiv[v]->vh = new double[cRnnCell->getANN(v)->getTopology()->getLayerSize(1)*M*M];
  }
}


void Rnn::resetHDerivatives(Derivatives** hderiv){
  for(int v = 0; v < V; v++){
    for(int k = 0; k < cRnnCell->getANN(v)->getTopology()->obtainWeightCount()*M; k++)
      hderiv[v]->v[k] = 0.0;

    for(int k = 0; k < cRnnCell->getANN(v)->getTopology()->getLayerSize(1)*M*M; k++)
      hderiv[v]->vh[k] = 0.0;

  }
}
void Rnn::resetCDerivatives(Derivatives** cderiv){
  for(int v = 0; v < V; v++){
    for(int k = 0; k < cRnnCell->getANN(v)->getTopology()->obtainWeightCount()*M; k++)
      cderiv[v]->v[k] = 0.0;

    for(int k = 0; k < cRnnCell->getANN(v)->getTopology()->getLayerSize(1)*M*M; k++)
      cderiv[v]->vh[k] = 0.0;

  }
}


void Rnn::copyVector(double* vec\_b, double *vec\_a, int n){
  for(int k = 0; k < n; k++)
    vec\_b[k] = vec\_a[k];
}

void Rnn::sumErrorDerivatives(double *h, Derivatives **hderiv, double *y){
  for(int v = 0; v < V; v++){
    Topology *top = cRnnCell->getANN(v)->getTopology();
    for(int s = 0; s < top->getLayerCount()-1; s++){
      for(int wi = 0; wi < top->getLayerSize(s)+1; wi++){
        for(int wj = 0; wj < top->getLayerSize(s+1); wj++){
          double sum = 0;
          for(int n = 0; n < M; n++)
            sum += (y[n]-h[n])*hderiv[v]->v[cRnnCell->getANN(v)->vi(v, s, wi, wj, n)];
          errDeriv[v]->v[cRnnCell->getANN(v)->vi(v, s, wi, wj)] = sum;
        }
      }
    }


    for(int wi = 0; wi < M; wi++){
      for(int wj = 0; wj < top->getLayerSize(1); wj++){
        double sum = 0;
        for(int n = 0; n < M; n++)
          sum += (y[n]-h[n])*hderiv[v]->vh[cRnnCell->getANN(v)->vhi(v, wi, wj, n)];
        errDeriv[v]->vh[cRnnCell->getANN(v)->vhi(v, wi, wj)] = sum;
      }
    }
  }
}

double Rnn::calcError(double *h, double *y){
  double error = 0;
  for(int k = 0; k < M; k++)
    error += 0.5*(h[k] - y[k])*(h[k] - y[k]);
  return error;
}

double f(double x){
  double y = 1 + exp(-x);
  return 1 / y;
}

double f\_deriv(double x){
  return exp(-x) / pow((1 + exp(-x)), 2);
}

double f\_tanh(double x){
  return tanh(x);
}

double f\_tanh\_deriv(double x){
  return 1 - pow(tanh(x), 2);
}


\subsection{app.h}


class LanguageModel {
  private:
  public:
    LanguageModel();

    void doSomething();

  private:
    std::vector<DataNode*>* loadFromFile(const char *abc, const char *filename);

};

double* char\_to\_vec(const char* abc, char c);
char vec\_to\_char(const char* abc, double  *vec);

DataNode* str\_to\_nodes(const char* abc, const char* str);
void nodes\_to\_str(const char* abc, DataNode* node, char* str);



\subsection{app.cpp}

LanguageModel::LanguageModel(){

}

void LanguageModel::doSomething(){

   char abc[64]=" abcdefghijklmnopqrstuvwxyz";
   // char abc[64]=" 123";
   // char abc[64]=" abcdeghilnoprstuvyz";

  //
  // int M = strlen(abc);
  // int I = strlen(abc);
  // printf("%s\n", "asdsa");
  // int V = 4;
  // Topology **topology = new Topology*[V];
  // for(int v = 0; v < V; v++){
  //   topology[v] = new Topology();
  //   topology[v]->addLayer(I);
  //   topology[v]->addLayer(M);
  // }
  //
  // RnnCell *rnnCell = new RnnCell(M, "500\_123\_03.bin");
  // Rnn* rnn = new Rnn(I, M, rnnCell, RNN\_FULL\_BACKPROPAGATION);
  //
  // SecondMarkLimit* markLimit = new SecondMarkLimit(0, M);
  //   // printf("%s\n", "asdsa");
  //     // printf("%s\n", "asdsa");
  // DataNode *inputNodes = str\_to\_nodes(abc, "123 4");
  //   // printf("%s\n", "asdsa");
  // DataNode* outputNodes = rnn->feedForward(inputNodes, markLimit);
  //   // printf("%s\n", "asdsa");
  // char str[512]="";
  // nodes\_to\_str(abc, outputNodes, str);
  //
  // printf("%s", str);
  // printf("%s\n", "asdsa");






  // std::vector<DataNode*>* nodeVector = loadFromFile(abc, "../files/data.txt");
  std::vector<DataNode*>* nodeVector = loadFromFile(abc, "data.txt");



  char str[512]="";
  for(int i = 0; i < nodeVector->size(); i++){
    nodes\_to\_str(abc, (*nodeVector)[i], str);

  }



  double alpha = 0.8;
  double eta = 0.05;

  int M = strlen(abc);
  int V = 4;
  int I = strlen(abc);
  Topology **topology = new Topology*[V];
  for(int v = 0; v < V; v++){
    topology[v] = new Topology();
    topology[v]->addLayer(I);
    topology[v]->addLayer(M);
  }

  RnnCell *rnnCell = new RnnCell(M, topology);
  Rnn *rnn = new Rnn(I, M, rnnCell, RNN\_APPROX\_BACKPROPAGATION);

  SecondMarkLimit* markLimit = new SecondMarkLimit(0, M);

  // printf("%s\n", "asdsa");
  double startTime = clock();

  for(int n = 0; n < 10; n++){
    double iterError = 0;
    for(int i = 0; i < nodeVector->size(); i++){
      DataNode* input = (*nodeVector)[i];


      DataNode* startOutput = input;

      for(int i = 0; i < 3; i++)
        startOutput = startOutput->next;


      DataNode* output = NULL;
      int offset = 0;

      int partCount = 0;
      double sentenceError = 0.0;

      do{
        double partError;
        output = startOutput;
        for(int i = 0; i < offset; i++)
          output = output->next;
        if(rnn->backPropagation(input, output, markLimit, partError) == false){
          rnn->resetErrorDerivatives();
          break;
        }
        rnn->updateWeights(alpha, eta);
        rnn->resetErrorDerivatives();

        sentenceError += partError;
        partCount++;


        offset++;
      }while(true);
      sentenceError = sentenceError / (double)partCount;

      iterError += sentenceError;
    }

     printf("Epochos paklaida: %.10f\n",iterError);
  }

  rnn->getRnnCell()->printf\_Network("2000\_123\_02.bin");



  double endTime = clock();
  double runtime = (double)(endTime-startTime)/CLOCKS\_PER\_SEC;

  printf("Apmokymas uztruko: %.5f sec\n", runtime);


}

std::vector<DataNode*>* LanguageModel::loadFromFile(const char *abc, const char *filename){
  FILE* file = fopen("../files/data.txt", "r");

  char* line = NULL;
  size\_t len = 0;
  ssize\_t read;

  std::vector<DataNode*>* vector = new std::vector<DataNode*>();

  while ((read = getline(\&line, \&len, file)) != -1) {
    printf("\\n\\n");
    printf("%s", line);

    DataNode* node = str\_to\_nodes(abc, line);
    vector->push\_back(node);


  }

  fclose(file);
  return vector;
}

//
// GLOBAL
//
void char\_to\_vec(const char* abc, double* vec, char c){
  int n = strlen(abc);


  for(int i = 0; i < n; i++)
    vec[i] = abc[i] == c ? 1.0 : 0.0;

}

char vec\_to\_char(const char* abc, double *vec){
  int n = strlen(abc);
  int maxAt = 0;
  for(int i = 1; i < n; i++)
    maxAt = vec[i] > vec[maxAt] ? i : maxAt;

  return abc[maxAt];
}

DataNode* str\_to\_nodes(const char* abc, const char* str){
  int n = strlen(abc);

  DataNode* out = new DataNode(n);
  DataNode* q = out;
  int index = 0;
  while(str[index] != '\\0'){
    char\_to\_vec(abc, q->vec, str[index]);
    index++;

    if(str[index] == '\\0') break;
    q->next = new DataNode(n);
    q = q->next;
  }

  return out;

}

void nodes\_to\_str(const char* abc, DataNode* node, char* str){
  int n = strlen(abc);
  DataNode* q = node;

  int index = 0;
  while(q != NULL){
    str[index++] = vec\_to\_char(abc, q->vec);
    q = q->next;
  }

  str[index] = '\\0';

}
