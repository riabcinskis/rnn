

\begin{titlepage}
  \begin{center}

  %  \vspace*{20pt}

    \begin{figure}[H]
      \centering
    \scalebox{1.0}{\includegraphics{images/NNwo6U.png}}
    \end{figure}

    \vspace*{30pt}

    \fontXII
    \textbf{Kauno technologijos universitetas}\\
    Matematikos ir gamtos mokslų fakultetas

    \vspace*{100pt}

    \fontXVIII
	   \textbf{Rekurentinio neuroninio tinklo apmokymas panaudojant grafinį vaizdo procesorių ir tinklo pritaikymas labiausiai tikėtinų žodžių pasiūlymui duotai sakinio pradžiai}\\
     \fontXIV
     Baigiamasis bakalauro projektas

     \vspace*{76.8pt}

     \fontXII

     % \begin{center}
     \singlespacing
     \begin{table}[H]
       \centering
      \begin{tabular}{C{90mm}}
      \arrayrulecolor[rgb]{0.831,0.686,0.216}\hline
      \begin{tabular}{@{}c@{}}~\\~\end{tabular} \\
      \begin{tabular}{@{}c@{}}\textbf{Deividas Riabčinskis}\\Projekto autorius\end{tabular} \\
      \begin{tabular}{@{}c@{}}~\\~\end{tabular} \\
      \begin{tabular}{@{}c@{}}\textbf{asist. Mindaugas Bražėnas}\\Vadovas\end{tabular} \\
        \begin{tabular}{@{}c@{}}~\\~\end{tabular} \\
        \begin{tabular}{@{}c@{}}\textbf{doc. vardas pav}\\Vadovas\end{tabular} \\
      \begin{tabular}{@{}c@{}}~\\~\end{tabular} \\ \arrayrulecolor[rgb]{0.831,0.686,0.216}\hline
      \end{tabular}
    \end{table}

\onehalfspacing

     \vfill

     \textbf{2019, Kaunas}

   \end{center}
\end{titlepage}
