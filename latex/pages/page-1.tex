

\begin{titlepage}
  \begin{center}

  %  \vspace*{20pt}

    \begin{figure}[H]
      \centering
    \scalebox{1.0}{\includegraphics{images/NNwo6U.png}}
    \end{figure}

    \vspace*{30pt}

    \fontXII
    \textbf{Kauno technologijos universitetas}\\
    Matematikos ir gamtos mokslų fakultetas\\
    Informatikos fakultetas

    \vspace*{80pt}

    \fontXVIII
	   \textbf{Rekurentinio neuroninio tinklo apmokymas panaudojant grafinį vaizdo procesorių ir tinklo pritaikymas labiausiai tikėtinų žodžių pasiūlymui duotai sakinio pradžiai}\\
     \fontXIV
     Baigiamasis bakalauro projektas\\

     %\vspace*{55pt}
     \vspace*{40pt}

     \fontXII



     \singlespacing
     \hfill
     \begin{table}[H]
      \begin{tabular}{@{}L{85mm}@{}@{}L{85mm}@{}}
      \arrayrulecolor[rgb]{0.831,0.686,0.216}\hline
      &  \\
      \begin{tabular}{@{}l@{}}\textbf{Deividas Riabčinskis}\\Projekto autorius\end{tabular} &  \\
      &  \\
      \arrayrulecolor[rgb]{0.831,0.686,0.216}\hline
      &  \\
      \begin{tabular}{@{}l@{}}\textbf{Taikomoji matematika (612G10002)}\\\textbf{pagrindinės krypties studijos}\end{tabular} & \begin{tabular}{@{}l@{}}\textbf{Informatika (612I10004)}\\\textbf{gretutinės krypties studijos}\end{tabular} \\
      &  \\
      \arrayrulecolor[rgb]{0.831,0.686,0.216}\hline
      &  \\
      \begin{tabular}{@{}l@{}}\textbf{asist.}\\\textbf{Mindaugas Bražėnas}\\Vadovas\end{tabular} & \begin{tabular}{@{}l@{}}\textbf{doc.}\\\textbf{Agnė Paulauskaitė-Tarasevičienė}\\Vadovė\end{tabular} \\
      & \\
      \begin{tabular}{@{}l@{}}\textbf{lekt. dr.}\\\textbf{Mantas Landauskas}\\Recenzentas\end{tabular} & \begin{tabular}{@{}l@{}}\textbf{doc.}\\\textbf{Martynas Patašius}\\Recenzentas\end{tabular} \\
      & \\ \arrayrulecolor[rgb]{0.831,0.686,0.216}\hline
      \end{tabular}
    \end{table}

\onehalfspacing

     \vfill

     \textbf{2019, Kaunas}

   \end{center}
\end{titlepage}
